% !TeX root = RJwrapper.tex
\title{R package downloads: what does it mean?}
\author{by Emi Tanaka}

\maketitle

\abstract{%
Abstract
}

\hypertarget{introduction}{%
\section{Introduction}\label{introduction}}

Today, R is greatly enhanced by over X R-packages contributed by X of
developers all over the world. However, when R originally appeared in
August of 1993 with its first official release in June of 1995 (Ihaka
1998), the contributions were managed by only a small group of core
developers. In April of 1997, the Comprehensive R Archive Network (CRAN)
was established as the official R-packages repository, with 3 mirror
sites. Now, the source repositories to install R-packages have expanded
to Bioconductor, Gitlab, GitHub, R-Forge and 106 CRAN mirrors in 49
regions. Of all the CRAN mirrors, the daily download counts for each
package is only readily available from the RStudio CRAN mirror.

\hypertarget{data}{%
\section{Data}\label{data}}

The main source of data used in this report is the download logs from
the RStudio CRAN mirror site : \url{https://cran.rstudio.com/}. These
log files are created for every instance of download of an R-package via
the RStudio CRAN mirror, then these log files are processed, daily, into
CSV files that contain the following variables with the name of header
in brackets:

\begin{itemize}
\tightlist
\item
  Date (\texttt{date}),
\item
  Time in UTC time zone (\texttt{time}),
\item
  Size of the file in bytes (\texttt{size}),
\item
  Version of R used to download the package (\texttt{r\_version}),
\item
  Architecture type for R (i386 = 32 bit, x86\_64 = 64 bit)
  (\texttt{r\_arch}),
\item
  Operating System (darwin9.8.0 = mac, mingw32 = windows)
  (\texttt{r\_os}),
\item
  Package (\texttt{package}),
\item
  Country in two letter ISO country code (\texttt{country}), and
\item
  Anonymised daily unique id (\texttt{ip\_id}).
\end{itemize}

A similar log file is also created for every download of R from the
RStudio CRAN mirror with the processed log file generating a CSV file
that contains the same variables except \texttt{r\_arch} and
\texttt{package}, and \texttt{r\_version} and \texttt{r\_os} are named
as \texttt{version} and \texttt{os}. These CSV files are hosted at
\url{http://cran-logs.rstudio.com/} and updated daily with data
available from 1st October 2012.

The log files of a particular day is processed and compressed into a
single CSV file of about 40 megabytes (file sizes of earlier years are
much smaller due to lower number of download logs). As there are over
700,000 CSV files, a simple estimate of the size of the data is 28
terabytes - far exceeding typical portable hard drives which are 1-4
terabytes.

The summarised version of data, where the data show the total daily
download counts for each package, is accessible using the
\texttt{cranlogs} R-package. The \texttt{cranlogs} package accesses this
summary data through the web application programming interface (API)
maintained by r-hub \citep{rhub}.

\hypertarget{results}{%
\section{Results}\label{results}}

\bibliography{../paper.bib}

\address{%
Emi Tanaka\\
Monash University\\%
Monash University\\ Clayton campus, VIC 3800, Australia\\
%
\url{http://emitanaka.org/}\\%
\textit{ORCiD: \href{https://orcid.org/0000-0002-1455-259X}{0000-0002-1455-259X}}\\%
\href{mailto:emi.tanaka@monash.edu}{\nolinkurl{emi.tanaka@monash.edu}}%
}

